% THIS IS SIGPROC-SP.TEX - VERSION 3.1
% WORKS WITH V3.2SP OF ACM_PROC_ARTICLE-SP.CLS
% APRIL 2009
%
% It is an example file showing how to use the 'acm_proc_article-sp.cls' V3.2SP
% LaTeX2e document class file for Conference Proceedings submissions.
% ----------------------------------------------------------------------------------------------------------------
% This .tex file (and associated .cls V3.2SP) *DOES NOT* produce:
%       1) The Permission Statement
%       2) The Conference (location) Info information
%       3) The Copyright Line with ACM data
%       4) Page numbering
% ---------------------------------------------------------------------------------------------------------------
% It is an example which *does* use the .bib file (from which the .bbl file
% is produced).
% REMEMBER HOWEVER: After having produced the .bbl file,
% and prior to final submission,
% you need to 'insert'  your .bbl file into your source .tex file so as to provide
% ONE 'self-contained' source file.
%
% Questions regarding SIGS should be sent to
% Adrienne Griscti ---> griscti@acm.org
%
% Questions/suggestions regarding the guidelines, .tex and .cls files, etc. to
% Gerald Murray ---> murray@hq.acm.org
%
% For tracking purposes - this is V3.1SP - APRIL 2009

\documentclass{acm_proc_article-sp}

\begin{document}

\title{Datanet}
\subtitle{Assignment 2}

% You need the command \numberofauthors to handle the 'placement
% and alignment' of the authors beneath the title.
%
% For aesthetic reasons, we recommend 'three authors at a time'
% i.e. three 'name/affiliation blocks' be placed beneath the title.
%
% NOTE: You are NOT restricted in how many 'rows' of
% "name/affiliations" may appear. We just ask that you restrict
% the number of 'columns' to three.
%
% Because of the available 'opening page real-estate'
% we ask you to refrain from putting more than six authors
% (two rows with three columns) beneath the article title.
% More than six makes the first-page appear very cluttered indeed.
%
% Use the \alignauthor commands to handle the names
% and affiliations for an 'aesthetic maximum' of six authors.
% Add names, affiliations, addresses for
% the seventh etc. author(s) as the argument for the
% \additionalauthors command.
% These 'additional authors' will be output/set for you
% without further effort on your part as the last section in
% the body of your article BEFORE References or any Appendices.

\numberofauthors{1}

\author{
% You can go ahead and credit any number of authors here,
% e.g. one 'row of three' or two rows (consisting of one row of three
% and a second row of one, two or three).
%
% The command \alignauthor (no curly braces needed) should
% precede each author name, affiliation/snail-mail address and
% e-mail address. Additionally, tag each line of
% affiliation/address with \affaddr, and tag the
% e-mail address with \email.
%
% 1st. author
\alignauthor
Troels Thomsen\\
       \affaddr{qvw203}\\
       \affaddr{Institute of Computer Science}\\
       \affaddr{Copenhagen University}
% use '\and' if you need 'another row' of author names
}

\maketitle

\begin{abstract}
In this assignment we will describe our simple webserver implementation, which currently only supports GET requests, ignoring all incoming headers.
We will see however, that our chosen structure is ideal for future improvement.
\end{abstract}

\keywords{Python, HTTP, request, socket, TCP, header} 

\section{Webserver design}
This sections describes the simply http server implementation.

\subsection{Structure}
\begin{itemize}
    \item Webserver:
        \begin{itemize}
        \item server:
        \begin{itemize}        
            \item HttpServer:
        \end{itemize}
        \item protocol:
        \begin{itemize}
            \item Protocol    
            \item HttpRequest
            \begin{itemize}
                \item HttpGet
            \end{itemize}
        \end{itemize}
        \item filesystem
        \begin{itemize}
            \item Filesystem
        \end{itemize}
        \item test
        \begin{itemize}
            \item TestProtocol
        \end{itemize}
    \end{itemize}
\end{itemize}

\subsection{Flow}
The webserver is structured into several classes. The main class being the HttpServer, which handles the incoming requests, filters the request type and delegates the headers to a child of the HttpRequest class.\\
In the current form our webserver only supports the GET method, so we delegate GET requests to the HttpGet class.\\

The HttpGet class then looks for the requested file or directory using the Filesystem class, and serves the appropriate file or directory listing back to the client using the Response class, which handles the formatting of the actual http response.

\section{Limitations}
Currently the http server only responds to GET requests. It also does not accept long paths for the get request, due to a flaw in the regular expression used to match the path and filename. This means that the server currently only supports accessing the root directory, however if the path was filtered correctly this would not be a problem.

The server also does not support any incoming headers. 

\section{Tests}
During the development we have mostly performed manual testing. We do have one unittest though, which tests for correct path parsing. This test currently fails due to a flaw in the regular expression.

\section{Supported Headers}
We ignore all incoming headers, and only serve the Content-Type and Content-Length header in our response.\\
These two headers pose the minimum, since they are required by most browsers.

\section{Extensibility}
The project structure allows for extensibility by design. We can for instance easily add support for POST and HEAD operations by creating HttpPost and HttpHead classes respectively.

We currently need to further improve the path matching, and add a unified header handling to the HttpRequest class. Besides these two points there should little need for further modification prior to adding functionality.

\section{Acknowledgements}
The webserver implementation was made in conjunction with Allan Nielsen (jcl187) and Troels Kamp Leskes (blr156). 

\end{document}

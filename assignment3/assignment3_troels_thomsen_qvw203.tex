% THIS IS SIGPROC-SP.TEX - VERSION 3.1
% WORKS WITH V3.2SP OF ACM_PROC_ARTICLE-SP.CLS
% APRIL 2009
%
% It is an example file showing how to use the 'acm_proc_article-sp.cls' V3.2SP
% LaTeX2e document class file for Conference Proceedings submissions.
% ----------------------------------------------------------------------------------------------------------------
% This .tex file (and associated .cls V3.2SP) *DOES NOT* produce:
%       1) The Permission Statement
%       2) The Conference (location) Info information
%       3) The Copyright Line with ACM data
%       4) Page numbering
% ---------------------------------------------------------------------------------------------------------------
% It is an example which *does* use the .bib file (from which the .bbl file
% is produced).
% REMEMBER HOWEVER: After having produced the .bbl file,
% and prior to final submission,
% you need to 'insert'  your .bbl file into your source .tex file so as to provide
% ONE 'self-contained' source file.
%
% Questions regarding SIGS should be sent to
% Adrienne Griscti ---> griscti@acm.org
%
% Questions/suggestions regarding the guidelines, .tex and .cls files, etc. to
% Gerald Murray ---> murray@hq.acm.org
%
% For tracking purposes - this is V3.1SP - APRIL 2009

\documentclass{acm_proc_article-sp}

\begin{document}

\title{Datanet}
\subtitle{Assignment 3}

% You need the command \numberofauthors to handle the 'placement
% and alignment' of the authors beneath the title.
%
% For aesthetic reasons, we recommend 'three authors at a time'
% i.e. three 'name/affiliation blocks' be placed beneath the title.
%
% NOTE: You are NOT restricted in how many 'rows' of
% "name/affiliations" may appear. We just ask that you restrict
% the number of 'columns' to three.
%
% Because of the available 'opening page real-estate'
% we ask you to refrain from putting more than six authors
% (two rows with three columns) beneath the article title.
% More than six makes the first-page appear very cluttered indeed.
%
% Use the \alignauthor commands to handle the names
% and affiliations for an 'aesthetic maximum' of six authors.
% Add names, affiliations, addresses for
% the seventh etc. author(s) as the argument for the
% \additionalauthors command.
% These 'additional authors' will be output/set for you
% without further effort on your part as the last section in
% the body of your article BEFORE References or any Appendices.

\numberofauthors{1}

\author{
% You can go ahead and credit any number of authors here,
% e.g. one 'row of three' or two rows (consisting of one row of three
% and a second row of one, two or three).
%
% The command \alignauthor (no curly braces needed) should
% precede each author name, affiliation/snail-mail address and
% e-mail address. Additionally, tag each line of
% affiliation/address with \affaddr, and tag the
% e-mail address with \email.
%
% 1st. author
\alignauthor
Troels Thomsen\\
       \affaddr{qvw203}\\
       \affaddr{Institute of Computer Science}\\
       \affaddr{Copenhagen University}
% use '\and' if you need 'another row' of author names
}

\maketitle

\begin{abstract}
In this assignment we will describe our simple webserver implementation, which currently only supports GET requests, ignoring all incoming headers.
We will see however, that our chosen structure is ideal for future improvement.
\end{abstract}

\keywords{Java, HTTP, request, response, socket, TCP, header, peer} 

\section{Theory}

\subsection{TCP as a choice for tracker communication}
We do not want to risk loosing any data transmitted by the tracker, and for this reason alone TCP is preferred since it guarantees data correctness.

The communication with the tracker is also done in small bursts over time, and thus the nature of UDP would not add any benefits for the client.

One could argue that a large tracker with thousands or millions of peers would benefit from using UDP, since each request would have much less overhead. In our case however, we feel that the correctness of data is more important.

\subsection{HTTP as a choice for tracker communication}
The amount of communication between the tracker and the client is relatively small. We only use the tracker for finding the peers associated with a given file. This means that the amount of traffic is limited to the initial POST request, and the occasional PUT request, when the peer updates its block string. Because of the small number of requests, we do not consider the overhead generally associated with HTTP a problem.

Futhermore the HTTP protocol is widely known and supported, and is thus much easier to implement in almost any language, since there exists countless HTTP libraries. 

Because of these two factors, there is no obvious reason to to choose HTTP.

\section{Experiments}
In the following experiments we will uncover the effectiveness of the kascade network.

\subsection{•}

\end{document}
